% This is "sig-alternate.tex" V2.1 April 2013
% This file should be compiled with V2.5 of "sig-alternate.cls" May 2012
%

\documentclass[english]{sig-alternate-05-2015}



\usepackage{babel}
\usepackage{csquotes}
\MakeOuterQuote{"}

%\usepackage[backend=biber,style=numeric-comp]{biblatex}
%\bibliography{references}
%\renewcommand{\bibliography}[1]{\printbibliography}

\usepackage{MnSymbol}

\usepackage[svgnames]{xcolor}
%\usepackage{todonotes}
%\presetkeys{todonotes}{inline}{}
\usepackage[subject={TODO}]{pdfcomment}
\newcommand{\todo}[2][]{\pdfmargincomment[author={#1}]{#2}}

%\usepackage{makecell}
\usepackage{ctable}


\usepackage{listings}

\newcommand{\lineref}[2]{\hyperref[#1]{line~\ref*{#1:#2}}}
\newcommand{\linerefn}[2]{\hyperref[#1]{line~#2}}
\newcommand{\linesrefn}[2]{\hyperref[#1]{lines~#2}}
\usepackage{accsupp}
\newcommand\emptyaccsupp[1]{\BeginAccSupp{ActualText={}}#1\EndAccSupp{}}

\usepackage{hyperref}
\usepackage[all]{hypcap}
\PassOptionsToPackage{nameinlink,noabbrev}{cleveref}
\usepackage{cleveref}


% \lstset{captionpos=b, xleftmargin=1cm}
% \lstset{basicstyle=\small\ttfamily, showstringspaces=false, columns=flexible, keepspaces=true}
\lstset{showstringspaces=false, columns=flexible, keepspaces=true}
\lstset{tabsize=2, gobble=2}
%\lstset{upquote=true}
%% Linebreaks
%\lstset{prebreak=\raisebox{0ex}[0ex][0ex]
%        {\ensuremath{\rhookswarrow}}}
\lstset{postbreak=\raisebox{0ex}[0ex][0ex]{\ensuremath{\rcurvearrowse\space}}}
\lstset{breaklines=true, breakatwhitespace=true}
%% Line Numbers
\newcommand{\lstnumberstyle}[1]{\tiny\emptyaccsupp{#1}}
\lstset{numbers=left, numberstyle=\lstnumberstyle, numbersep=5pt,
		numberfirstline=true, firstnumber=1, stepnumber=5}
\lstset{escapeinside={(*@}{@*)}}
\lstdefinestyle{BWStyle}{
	keywordstyle=\bfseries,
	stringstyle=\color{DimGray},
	commentstyle=\textsl,
}
\lstset{style=BWStyle}

\lstdefinelanguage{algorithm}{
	keywords={function, for, do, if, then, else, return, in_, is_a, or, and},
	morecomment=[l]{'},
	morecomment=[s]{/*}{*/},
	morestring=[b]",
	sensitive=true,
}
\lstdefinelanguage{HanaSQL}[]{SQL}{
	morekeywords={replace,string,if,is,daysbetween,secondsbetween,weekday,adddays,addseconds,double},
	moredelim=**[is][\slshape]{^}{^},
	moredelim=**[is][\bfseries]{§}{§},
}
\lstdefinelanguage{Inline}{
	moredelim=**[is][\slshape]{^}{^},
	moredelim=**[is][\bfseries]{§}{§},
}

\lstset{language=HanaSQL}
\lstMakeShortInline[basicstyle=\ttfamily,language={Inline},breaklines=true]°


\begin{document}

% Copyright
\setcopyright{acmcopyright}
%\setcopyright{acmlicensed}
%\setcopyright{rightsretained}
%\setcopyright{usgov}
%\setcopyright{usgovmixed}
%\setcopyright{cagov}
%\setcopyright{cagovmixed}


% DOI
\doi{10.475/123_4}

% ISBN
\isbn{123-4567-24-567/08/06}

%Conference
\conferenceinfo{PLDI '13}{June 16--19, 2013, Seattle, WA, USA}

\acmPrice{\$15.00}

%
% --- Author Metadata here ---
\conferenceinfo{WOODSTOCK}{'97 El Paso, Texas USA}
%\CopyrightYear{2007} % Allows default copyright year (20XX) to be over-ridden - IF NEED BE.
%\crdata{0-12345-67-8/90/01}  % Allows default copyright data (0-89791-88-6/97/05) to be over-ridden - IF NEED BE.
% --- End of Author Metadata ---

\title{TARDISP: Bringing back-in-time debugging to the database}
\subtitle{(TARDISP is a placeholder for the actual name)}
%
% You need the command \numberofauthors to handle the 'placement
% and alignment' of the authors beneath the title.
%
% For aesthetic reasons, we recommend 'three authors at a time'
% i.e. three 'name/affiliation blocks' be placed beneath the title.
%
% NOTE: You are NOT restricted in how many 'rows' of
% "name/affiliations" may appear. We just ask that you restrict
% the number of 'columns' to three.
%
% Because of the available 'opening page real-estate'
% we ask you to refrain from putting more than six authors
% (two rows with three columns) beneath the article title.
% More than six makes the first-page appear very cluttered indeed.
%
% Use the \alignauthor commands to handle the names
% and affiliations for an 'aesthetic maximum' of six authors.
% Add names, affiliations, addresses for
% the seventh etc. author(s) as the argument for the
% \additionalauthors command.
% These 'additional authors' will be output/set for you
% without further effort on your part as the last section in
% the body of your article BEFORE References or any Appendices.

\numberofauthors{3} 
\author{
% 1st. author
\alignauthor
Ben Trovato\titlenote{Dr.~Trovato insisted his name be first.}\\
       \affaddr{Institute for Clarity in Documentation}\\
       \affaddr{1932 Wallamaloo Lane}\\
       \affaddr{Wallamaloo, New Zealand}\\
       \email{trovato@corporation.com}
% 2nd. author
\alignauthor
G.K.M. Tobin\titlenote{The secretary disavows
any knowledge of this author's actions.}\\
       \affaddr{Institute for Clarity in Documentation}\\
       \affaddr{P.O. Box 1212}\\
       \affaddr{Dublin, Ohio 43017-6221}\\
       \email{webmaster@marysville-ohio.com}
% 3rd. author
\alignauthor Lars Th{\o}rv{\"a}ld\titlenote{This author is the
one who did all the really hard work.}\\
       \affaddr{The Th{\o}rv{\"a}ld Group}\\
       \affaddr{1 Th{\o}rv{\"a}ld Circle}\\
       \affaddr{Hekla, Iceland}\\
       \email{larst@affiliation.org}
%\and  % use '\and' if you need 'another row' of author names
}

%\date{30 July 1999}

\maketitle
\begin{abstract}
  Omniscient debuggers allow developers to explore what happened before a failure and so supports them in finding failure causes more efficiently. 
	We present an approach for bringing omniscient debugging to stored procedures that allows to step backwards and query the database at previous points in time.
  Furthermore, we introduce an extension to SQL that allows to express queries covering a period of execution time within a debug session.
	Our prototype implementation is able to debug procedures handling large amounts of data with little overhead on performance and memory.
\end{abstract}

%
% The code below should be generated by the tool at
% http://dl.acm.org/ccs.cfm
% Please copy and paste the code instead of the example below. 
%
\begin{CCSXML}
<ccs2012>
 <concept>
  <concept_id>10010520.10010553.10010562</concept_id>
  <concept_desc>Computer systems organization~Embedded systems</concept_desc>
  <concept_significance>500</concept_significance>
 </concept>
 <concept>
  <concept_id>10010520.10010575.10010755</concept_id>
  <concept_desc>Computer systems organization~Redundancy</concept_desc>
  <concept_significance>300</concept_significance>
 </concept>
 <concept>
  <concept_id>10010520.10010553.10010554</concept_id>
  <concept_desc>Computer systems organization~Robotics</concept_desc>
  <concept_significance>100</concept_significance>
 </concept>
 <concept>
  <concept_id>10003033.10003083.10003095</concept_id>
  <concept_desc>Networks~Network reliability</concept_desc>
  <concept_significance>100</concept_significance>
 </concept>
</ccs2012>  
\end{CCSXML}

\ccsdesc[500]{Computer systems organization~Embedded systems}
\ccsdesc[300]{Computer systems organization~Redundancy}
\ccsdesc{Computer systems organization~Robotics}
\ccsdesc[100]{Networks~Network reliability}

\printccsdesc

\keywords{ACM proceedings; \LaTeX; text tagging}

\section{Introduction}

Finding defects in code is a frequent task for every programmer and is often difficult even with a deep understanding of the system.
Back-in-time debuggers simplify the debugging process \todo{cite} by making it easier to follow causal chains from the failure to the defect.

With the rise of big-data applications, performance requirements demand code to move closer to the database\todo{cite}.
The increasing complexity of application code in the database creates a greater need for better debugging tools.
Alas, even though back-in-time debuggers exist for many object-oriented programming languages\todo{cite}, there are none that run on databases and support SQL or SQL Script.

Back-in-time debuggers typically create a significant overhead on performance and memory consumption.
Thus, it seems unfeasible to use a back-in-time debugger on top of a database script that processes billions of records.
Nevertheless, such scripts need to be debugged and, in particular for scripts with writing operations, the existing tools are often not adequate for this task.
Better tools can speed-up develoment and reduce maintenance cost of database scripts.

In this paper, we bring the concept of back-in-time debugging to the database and present \emph{TARDISP}, an implementation of our approach.
The contributions of this paper are as follows:
\begin{itemize}
	\item \emph{TARDISP} is an omniscient debugger for stored procedures which can be installed in the SAP HANA in-memory database and programming platform.
		Using TARDISP, developers can move freely through the execution time of a stored procedure and inspect control flow, variables, and intermediate results.
	
	\item \emph{TARDISQL} is an extension to SQL allowing to submit arbitrary queries against previous states of the database 
		and to compare multiple points in time with one query.
		The TARDISP provides a console for developers to submit TARDISQL queries which use variables or points in time from the current debug session.

	\item \emph{Very low overhead} when recording run-time data and the efficient querying of past database states allow developers to use TARDISP as the default tool for debugging database scripts even on larger data sets.
		First interviews with developers have indicated that bugs in stored procedures can be investigated more efficiently with TARDISP than with a regular debugger.
	
\end{itemize}

The remainder of this paper is structured as follows:
Our approach and our prototype implementation is described in \cref{sec:prototype}.
In \cref{sec:ttqueries}, we explain back-in-time queries.



\section{Related Work}

\section{Omniscient Debugging}
\label{sec:prototype}

- typical problem: much data \\
- backend: HANA xs-engine \\
- frontend: HTML5 \\


\subsection{Data Model}

A typical limitation for omniscient debuggers is the large amount of trace data generated when processing large data sets, which typically exceeds the program data by orders of magnitude.
Since our debugger is supposed to work with large databases, using an even bigger database just to manage debug sessions is not feasible.
Instead, we take advantage of the fact that all instructions handling large data sets, i.e., SQL statements, are declarative and reproducible as long as the underlying data does not change.
%Thus, SQL queries don't have to be traced at all, although for some purposes it will be helpful to record some meta information, such as the execution time or the number of results.

To reproduce the execution of a stored procedure, we need to trace a sequence of execution steps.
Each step represents an instruction that has side effects or returns some result to the procedure.
As there is, conceptually, no concurrency in a stored procedure, we use simple numbering to track the order of steps.
All trace data is stored in a relational database.

The debugger distinguishes between 8 types of steps:
three types of control flow steps, for calling, entering and exiting stored procedures;
four kinds of variable assignments for atomic values, query results, cursors, and cursor rows;
and statements with side-effects.
If applicable, the step has a target name, i.e. the name of a variable or stored procedure, and a string representation of the value, which is shown to the user in the debug view.

If the step involves a database statement, we don't trace any details of the execution but only record the name of a view than can be used to later reproduce the query result.
Additionally, we store the number of rows in the result set and timestamps from before and after the execution.

%- Traces \\
%- Steps \\
%- Queries \\
%- QueryArgs \\

\subsection{Tracing}

With a debugger fully integrated in its database, we would expect the stored procedure execution engine to trace the execution steps.
However, for our prototype this level of integration was out of scope.

Instead, we wrote a pre-processor in Java that parses a stored procedure and inserts °INSERT° statements around every instruction to collect the required trace data.
To obtain a trace, the debugger then has to run the traced code once instead of the original procedure.

In addition to the tracing code, the pre-processor also generates SQL functions that will be used to obtain variable values at given points in time.
For atomic variables, it is simply a search for its most recent step.

For variables containing result tables, a separate function is generated for each query that assigns to that variable.
Each function contains the original code of its query.
Variables used as parameters by the query are initialized using their respective value functions.
On top of that, a master function is generated that chooses the appropriate query function based on the requested step and the recorded step data.

\todo{example}
\todo{also: generates functions/views for variables}

%% ~~~~~~~~~~~~~~~~~~~~~~~~~~~~~~~~~~~~~~~~~~~~~~~~~
%% ~~~~~~~~~~~~~~~~~~~~~~~~~~~~~~~~~~~~~~~~~~~~~~~~~

\subsection{Reproducing Query Results}

With the recorded trace data, the debugger has enough information available to re-execute any query.
However, the query will only yield the same results as long as the underlying data has not changed.

In general, one can expect that debugging will take place on a development machine where no other data manipulation occurs.
However, in cases where this assumption doesn't hold, the debugger might end up showing wrong or misleading data the developer.
Furthermore, the debugged stored procedure itself may change the data, which will cause a query to return different results at different points in time.

We ensure consistency over time by following the insert-only approach of our in-memory database.
If we require for all tables that data can never be changed or deleted and annotate all tuples with timestamps of when they have been created and invalidated, we can reconstruct the state of the database of any point in time.
Adding timestamp filters to select queries does not cause a significant slowdown.
Our prototype was built using this approach.

\todo{HANA hat seit neustem "History tables" die das direkt können, wir benutzen aber explizite timestamps weil das momentan noch praktischer ist }


\section{Time-travel Queries}
\label{sec:ttqueries}

In our set-up, the debugger trying to recreate intermediate results of a stored procedure is just a special use case for the ability to submit arbitrary queries against the database of any previous point in time.

\subsection{The Step Concept}

The query shown in \cref{lst:ttravel} selects the total of open orders for previously selected projects.
We will use it as an example to demonstrate how \emph{time-traveling} queries are handled by our system.

\begin{lstlisting}[language=HanaSQL,float,caption={Example for a time-travel query: select the current total of open orders for previously selected projects},label=lst:ttravel]
  SELECT pr.id, pr.name, pr.budget, SUM(po.total)
  FROM :selected_projects pr
	JOIN PurchaseOrders po ON po.project_id = pr.id
	WHERE po.status = 'open'
	GROUP BY pr.id, pr.name, pr.budget
	^§AT STEP§ 1623^
\end{lstlisting}

The last line shows an extension to SQL that can be used by the developer to explicitly query a point in time, with °^1623^° being a step ID that was obtained from the debugger UI.
If omitted, the current step can be derived from the context from which the query is submitted, such as an SQL console that is associated with a specific point in time or the current debug step.
The parameter °:selected_projects° refers to a variable from the current debug session and will be populated with its current value, independently of the value of the step-clause.

When submitted, our debugger applies two changes to the query before it can be submitted to the database.
First, all variables are replaced with corresponding functions or views.
For variables containing atomic values, a function is generated that returns the variable's value at a given step.
For variables containing query results, a view is generated for each query.
These views are identified by the target variable name and the line number and expect a step number and all parameters that the actual query took.
\todo{explain variable functions}
In our example, °:selected_projects° might be replaced with °VAR_selected_projects_7(1055, 'Research')° when it was last set at step 1055 in code line 7 and called with the respective argument.

Second, a time-stamp filter is added for all tables that are referenced in the query. 
In our example,
\begin{lstlisting}[language={Inline},basicstyle=\ttfamily,numbers=none]
  po.createdOn < ^1623^ AND (po.validTo IS NULL OR po.validTo > ^1623^)
\end{lstlisting}
would be added to the Where-clause.

Now, the query can be submitted to the database and the result is subsequently presented to the user.

\subsection{Time-diff Queries}

\newcommand{\red}[1]{\textcolor{DarkRed}{#1}}
\newcommand{\gr}[1]{\textcolor{Green}{#1}}


\ctable[star,caption={Result of a time-diff query, with multiple values in some columns},label=tab:diffresult,doinside={}]
				{rlrrrlr}{}{
	pr.id & pr.name 	& pr.budget & total & po2.id & po2.status & po2.total \ML
	
				&					 & \red{1200} & \red{1500} &	 & \red{open} &						\NN
	1			& Project 1 & 200				& 500 			 & 1 & paid 			& 1000			\NN
				&						& \gr{-300} & \gr{0}		 &	 &						&						\ML

				&					 & \red{1200} & \red{1500} &	 & 						&						\NN
	1			& Project 1 & 200				& 500 			 & 2 & open 			& 500				\NN
				&						& \gr{-300} & \gr{0}		 &	 & \gr{paid}	&						\ML
}

To get a better overview about what happened in a piece of code, the developer might want to query multiple points in time at once and see the difference in the query result.
For this example, she debugs a stored procedure that processes the payments for projects, but sometimes allows projects to go over budget.
By stepping into the procedure, she has three defined points in time: °^before^°, at the beginning of the procedure; °^now^°, at the current instruction; and °^after^°, at the end of the execution.

Now she wants to compose a query that selects all projects that will go over budget and the orders that were processed.
\todo{explain query rewriting}
The query is shown in \cref{lst:tdiff}.
\begin{lstlisting}[language=HanaSQL,float=b,caption={Example of a time-diff query: "Select all projects that will go over budget and their respective purchase orders"},label=lst:tdiff]
	SELECT pr.id, pr.name, pr.budget, SUM(po.total), po2.id, po2.status, po2.total
	FROM :selectedProjects pr
	JOIN PurchaseOrders po ON po.project_id = pr.id
	JOIN PurchaseOrders po2 ON po2.project_id = pr.id
	WHERE po.status = 'open'
		AND now!pr.budget > 0 AND after!pr.budget < 0
		AND before!po2.status != after!po2.status
	GROUP BY pr.id, pr.name, pr.budget, po2.id, po2.status, po2.total
	^§AT STEP§ before=817, now=1623, after=2043^
\end{lstlisting}
Like before, the °^AT STEP^° clause does not have to be explicitly typed in the query, but can also be derived from the context.
A language extension allows to add filter conditions that only apply to specific points in time.
\Cref{tab:diffresult} shows a possible result for this query, with one project that goes over budget and two associated purchases, of which one was already processed.

To produce this result, the query has to be executed three times, once for each point in time, without the time-specific filter conditions.
Then, to prepare the diffing of the results, they are outer-joined on the primary keys and the time-specific filters are applied.
For performance reasons, all of this happens inside a single SQL query, as shown in \cref{lst:tdifffinal}.
The execution of the sub-queries is indicated in \linerefn{lst:tdifffinal}{6, 7, and 10}, the time-specific filters can be found in the Where-condition of \linerefn{lst:tdifffinal}{15 and 16}.

\begin{lstlisting}[language=HanaSQL,float,caption={Parts of the time-diff query after transformation},label=lst:tdifffinal]
	SELECT COALESCE(__before."pr.id", ...) AS "pr.id",
	       COALESCE(__before."po.id", ...) AS "po.id",
				 __before.createdOn as _step_0,
				 __before."pr_name" AS "pr_name_0", ...,
				 ...
	FROM (SELECT ... ^§AT STEP§ before^) __before
	FULL OUTER JOIN (SELECT ... ^§AT STEP§ now^) __now
	    ON __before."pr.id" = __now."pr.id" 
		 AND __before."po.id" = __now."po.id"
	FULL OUTER JOIN (SELECT ... ^§AT STEP§ after^) __after
	    ON (__before."pr.id" = __after."pr.id" 
		      AND __before."po.id" = __after."po.id")
		  OR (__now."pr.id" = __after."pr.id" 
		      AND __now."po.id" = __after."po.id")
	WHERE __now."pr.budget" > 0 AND __after."pr.budget" < 0
	  AND __before."po.status" != __after."po.status"
\end{lstlisting}

For the final result, the key attributes are coalesced while the other attributes are selected from each point in time.
Furthermore, for each tuple its creation step is selected.
This value is needed for two reasons: first, it is necessary to distinguish between tuples with °NULL° values and tuples completely missing from the result; second, it allows the debugger to know when the value was created or changed.

In the UI, the before and after values are only shown if they differ from the now value.
Clicking on value allows the developer to jump to the °UPDATE° or °INSERT° statement that caused the change.


\section{Evaluation}

- Test system: 2 billion records of point of sales data \\
- Stored procedure: calculate revenue and margin per week of current and last year \\
- Exec time: approx. 3 seconds; with tracing: no measurable overhead \\
- °SELECT * FROM :result°: 1.5 seconds \\
- \todo{time-diff queries}?

% bug-tracker 

\subsection{Limitations}

Currently, our approach has two major limitations.

First, time-diff queries can only be executed on tables that have clearly defined primary keys, for key attributes are required to track a tuple's versions over time.b
For a query like "Sum budgets per project category", it has to be clear that categories are the entities that keep their identity over time.
Here, an additional syntax extension could be used to convey this kind of information.

Second, it is currently not possible to use time qualifiers outside of the °WHERE° clause.


\section{Conclusion}



Future work

%
% The following two commands are all you need in the
% initial runs of your .tex file to
% produce the bibliography for the citations in your paper.
\bibliographystyle{abbrv}
\bibliography{sigproc}  % sigproc.bib is the name of the Bibliography in this case

\end{document}
